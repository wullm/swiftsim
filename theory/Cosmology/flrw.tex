\subsection{Background evolution}
\label{ssec:flrw}

In \swift we assume a standard FLRW metric for the evolution of the background
density of the Universe and use the Friedmann equations to describe the
evolution of the scale-factor $a(t)$.  We scale $a$ such that its present-day
value is $a_0 \equiv a(t=t_{\rm now}) = 1$. We also define redshift $z \equiv
1/a - 1$ and the Hubble parameter
\begin{equation}
H(t) \equiv \frac{\dot{a}(t)}{a(t)}
\end{equation}
with its present-day value denoted as $H_0 = H(t=t_{\rm now})$. Following
normal conventions we write $H_0 = 100
h~\rm{km}\cdot\rm{s}^{-1}\cdot\rm{Mpc}^{-1}$ and use $h$ as the input parameter
for the Hubble constant.

To allow for general expansion histories we use the full Friedmann equations
and write
\begin{align}
H(a) &\equiv H_0 E(a) \\ E(a) &\equiv\sqrt{\Omega_m a^{-3} + \Omega_r
  a^{-4} + \Omega_k a^{-2} + \Omega_\Lambda \exp\left(3\tilde{w}(a)\right)}, \label{eq:Ea}
\\
\tilde{w}(a) &= (a-1)w_a - (1+w_0 + w_a)\log\left(a\right),
\label{eq:friedmann}
\end{align}
where we followed \cite{Linder2003} to parametrize the evolution of
the dark-energy equation-of-state\footnote{Note that $\tilde{w}(z)\equiv
  \int_0^z \frac{1+w(z')}{1+z'}dz'$, which leads to the analytic
  expression we use.} as:
\begin{equation}
w(a) \equiv w_0 + w_a~(1-a).
\end{equation}
The cosmological model is hence fully defined by specifying the dimensionless
constants $\Omega_m$, $\Omega_r$, $\Omega_k$, $\Omega_\Lambda$, $h$, $w_0$ and
$w_a$ as well as the starting redshift (or scale-factor of the simulation)
$a_{\rm start}$ and final time $a_{\rm end}$. \\ At any scale-factor $a_{\rm
age}$, the time $t_{\rm age}$ since the Big Bang (age of the Universe) can be
computed as \citep[e.g.][]{Wright2006}:
\begin{equation}
  t_{\rm age} = \int_{0}^{a_{\rm age}} dt = \int_{0}^{a_{\rm age}}
  \frac{da}{a H(a)} = \frac{1}{H_0} \int_{0}^{a_{\rm age}}
  \frac{da}{a E(a)}. \label{eq:flrw:age}
\end{equation}
For a general set of cosmological parameters, this integral can only be
evaluated numerically, which is too slow to be evaluated accurately during a
run. At the start of the simulation we tabulate this integral for $10^4$ values
of $a_{\rm age}$ equally spaced between $\log(a_{\rm start})$ and $\log(a_{\rm
end})$. The values are obtained via adaptive quadrature using the 61-points
Gauss-Konrod rule implemented in the {\sc gsl} library \citep{GSL} with a
relative error limit of $\epsilon=10^{-10}$. The value for a specific $a$ (over
the course of a simulation run) is then obtained by linear interpolation of the
table.

\subsubsection{Additional quantities}

\swift computes additional quantities that enter common expressions appearing in
cosmological simulations. For completeness we give their expressions here. The
look-back time is defined as
\begin{equation}
  t_{\rm lookback} = \int_{a_{\rm lookback}}^{a_0} \frac{da}{a E(a)},
\end{equation}
and the critical density for closure at a given redshift as
\begin{equation}
  \rho_{\rm crit} = \frac{3H(a)^2}{8\pi G_{\rm{N}}}.
\end{equation}
These quantities are computed every time-step.

\subsubsection{Case of flat radiation-free Universe}

In the case where $\Omega_k = \Omega_r = 0$, $w_0=-1$ and $w_a = 0$, the
expression for $E(a)$ becomes a lot simpler and the time since the Big
Bang (\ref{eq:flrw:age}) can be computed analytically:

\begin{align}
  t_{\rm age} &= \frac{1}{H_0}\frac{2}{3\sqrt{\Omega_\Lambda}} \sinh^{-1}\left[\sqrt{\frac{a_{\rm age}^3 \Omega_\Lambda}{\Omega_m}}\right] \\
              &= \frac{1}{H_0}\frac{2}{3\sqrt{\Omega_\Lambda}}
\log\left[\sqrt{1+\frac{a_{\rm age}^3 \Omega_\Lambda}{\Omega_m}} + \sqrt{\frac{a_{\rm age}^3 \Omega_\Lambda}{\Omega_m}}\right].
\end{align}
We use this expression to test the accuracy of the interpolation
tables used for the more general case.

\subsubsection{Typical Values of the Cosmological Parameters}

Typical values for the constants are: $\Omega_m = 0.3, \Omega_\Lambda=0.7, 0 <
\Omega_r<10^{-3}, |\Omega_k | < 10^{-2}, h=0.7, a_{\rm start} = 10^{-2}, a_{\rm
end} = 1, w_0 = -1\pm 0.1, w_a=0\pm0.2$ and $\gamma = 5/3$.

\subsubsection{Massive neutrinos}

The non-relativistic transition of massive neutrinos occurs at
$a^{-1}\approx 1890 (m_\nu/1\text{ eV})$, which typically is in the timeframe of
cosmological simulations. This changes the background evolution slightly,
affecting the Hubble rate $E(a)$ and most integrated quantities described in this
document by a few tenths of a percent. To account for this effect, we allow users
to specify the photon density $\Omega_\gamma$ (or $T_\text{CMB}$), the number of
neutrino species $N_\nu$, the neutrino temperature $T_{\nu,0}$ (or $N_\text{eff}$)
and neutrino masses $m_{\nu,i}$. These quantities replace the total radiation
density parameter $\Omega_r$. The radiation term appearing in (\ref{eq:Ea}) is
replaced by
\begin{align}
    \Omega_r a^{-4} = \left[\Omega_\gamma + \Omega_\nu(a)\right] a^{-4}.
\end{align}
In the massless case $m_{\nu,i}=0$ for $i=1,\dots,N_\nu$, the neutrino density is constant and given by
\begin{align}
    \Omega_\nu = \frac{7}{8}\left(\frac{4}{11}\right)^{4/3} N_\text{eff}\,\Omega_\gamma.
\end{align}
In general, the neutrino density parameter is \citep{Zennaro2016}
\begin{align}
    \Omega_\nu(a) = \Omega_\gamma \sum_{i=1}^{N_\nu}\frac{15}{\pi^4}\left(\frac{T_{\nu,0}}{T_\text{CMB}}\right)^4 \mathcal{F}\left(\frac{a m_{\nu,i}}{k_b T_{\nu,0}}\right), \label{eq:nudensity}
\end{align}
where $k_b$ is Boltzmann's constant and the function $\mathcal{F}$ is given by the momentum integral
%
\begin{align}
    \mathcal{F}(y) = \int_0^{\infty} \frac{x^2\sqrt{x^2+y^2}}{1+e^{x}}\mathrm{d}x.
\end{align}
At the start of the simulation, but only if necessary, values of
(\ref{eq:nudensity}) are tabulated on a finer grid of $10^5$ values of $a$ spaced
between $\log(a_\text{begin})$ and $\log(a_\text{end})$. The integrals
$\mathcal{F}(y)$ are evaluated using the 61-points Gauss-Konrod rule implemented
in the {\sc gsl} library with a relative error limit of $\epsilon=10^{-12}$.
Tabulated values are then linearly interpolated whenever $E(a)$ is computed.

As indicated above, users have some flexibility with regard to input parameters
that are related. If one of $\Omega_\gamma$/$T_\text{CMB}$ and $N_\text{eff}$/$T_{\nu,0}$
is specified, the others are inferred from:
%
\begin{align}
    \Omega_\gamma &= \frac{\pi^2}{15}\frac{(k_b T_\text{CMB})^4}{(\hbar c)^3}\frac{1}{\rho_{\rm crit}c^2},\\
    N_\text{eff} &= N_\nu\left(\frac{11}{4}\right)^{4/3}\left(\frac{T_{\nu,0}}{T_\text{CMB}}\right)^4.
\end{align}
